%%TITRE
\begin{center}
    \vspace{0.5cm}
    \textbf{\Large JENNRICH'S ALGORITHM - A MATLAB IMPLEMENTATION} \\
    \vspace{0.5cm}
    CHAU Dang Minh
\end{center}

\section{Jennrich's Algorithm}
Let there be a three-order tensor $\X\in\RR^{m\times n\times p}$. Suppose that $\X$ has a canonical polyadic decomposition
\begin{equation}
    \X = [[\mathbf{A}, \mathbf{B}, \mathbf{C}]] = \sum\limits_{i=1}^{r} \mathbf{a}_i\otimes \mathbf{b}_i\otimes \mathbf{c}_i,
\end{equation}

where $\mathbf{A}\in\RR^{m\times r}$, $\mathbf{B}\in\RR^{n\mathbf{C} r}$  and $\mathbf{C}\in\RR^{p\times r}$. Choose a unit vector $\mathbf{x}_i\in\RR^p$ uniformly. We have

\begin{equation}
    \mathbf{M}_x = \sum\limits_{i=1}^{p} a_i\X_{:,:,i} = \sum\limits_{j=1}^{r} \langle \mathbf{c}_j, \mathbf{x} \rangle \mathbf{a}_j\mathbf{b}_j^\top = \mathbf{A}\mathrm{diag}(\langle \mathbf{c}_1, \mathbf{x} \rangle, \ldots, \langle \mathbf{c}_r, \mathbf{x} \rangle) \mathbf{B}^\top.
\end{equation}

Let $\mathbf{D}_x = \mathrm{diag}(\langle \mathbf{c}_1, \mathbf{x} \rangle, \ldots, \langle \mathbf{c}_r, \mathbf{x} \rangle)$ for brevity. We write

\begin{equation}
    \mathbf{M}_x = \mathbf{A} \mathbf{D}_x \mathbf{B}^\top.
\end{equation}

Similarly, we can choose a unit vector $\mathbf{b}$ uniformly and construct the matrix

\begin{equation}
    \mathbf{M}_y = \mathbf{A} \mathbf{D}_y \mathbf{B}^\top.
\end{equation}

If $\mathbf{B}$ is full column rank, we have $\mathbf{B}^\top (\mathbf{B}^\top)^\dag = \mathbf{I}_r$. Therefore,

\begin{equation}
    \mathbf{M}_x\mathbf{M}_y^\dag = \mathbf{A} \mathbf{D}_x \mathbf{B}^\top (\mathbf{B}^\top)^\dag \mathbf{D}_y^\dag \mathbf{A}^\dag = \mathbf{A}(\mathbf{D}_x \mathbf{D}_y^\dag) \mathbf{A}^\dag,
\end{equation}

where $\mathbf{D}_x \mathbf{D}_y^\dag = \mathrm{diag}\left(\dfrac{\langle \mathbf{C}_1, \mathbf{a} \rangle}{\langle \mathbf{C}_1, \mathbf{b} \rangle},\ldots, \dfrac{\langle \mathbf{C}_r, \mathbf{a} \rangle}{\langle \mathbf{C}_r, \mathbf{b} \rangle}\right)$. Since $\mathbf{x}$ and $\mathbf{y}$ are chosen uniformly, the elements of $\mathbf{D}_x \mathbf{D}_y^\dag$ are distinct with probability $1$. Hence, the columns of $\mathbf{A}$ are eigenvectors of $\mathbf{M}_x\mathbf{M}_y^\dag$. Similarly, if $\mathbf{A}$ is full column rank, then

\begin{equation}
    \mathbf{M}_x^\top (\mathbf{M}_y^\top)^\dag = \mathbf{B}(\mathbf{D}_x \mathbf{D}_y^\dag) \mathbf{B}^\dag,
\end{equation}

which means that the columns of $\mathbf{B}$ are eigenvectors of $\mathbf{M}_x^\top (\mathbf{M}_y^\top)^\dag$. Finally, we recover $\mathbf{C}$ using

\begin{equation}
    \X_{(3)} = \mathbf{C}(\mathbf{B}\odot \mathbf{A})^\top.
\end{equation}
